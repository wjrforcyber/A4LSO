% !TEX root = ../main.tex
\documentclass[main.tex]{subfiles}
\begin{document}

\section{Introduction}
Have a clear overview of the whole design flow helps in comprehending what happens under the hood and the reason why each step is needed.

\subsection{Concepts}
Some basic concepts or abbreviations you should know even after the introduction. Always check the concepts of the core idea in this course, it helps you understand in general.
\begin{problem}{}{problem-label}
\begin{enumerate}[(a)]
    \item What are the following terms abbreviated for:\\
    EDA\\
    VLSI\\
    FPGA
    \item Draw a whole design-flow flow chart, try to draw it as detailed as possible.\\
    Check Figure 1.1 in Yosys manual \cite{Yosys}, compare with what you have drawn.
    \item Can you map the idea to the language compiler?
\end{enumerate}
\end{problem}
\vspace*{4\baselineskip}

\subsection{Explore}
Explore the active research, have your own way of finding basic information will help you solve problems by yourself, always try to find solutions by yourself, note that there is a very high possiblity that there are prople already asked or illustrated what you want to know on the internet.
\begin{problem}{}{problem-label}
\begin{enumerate}[(a)]
    \item Do you know any of the active open source project in EDA area?
    \item Do you know where and how to efficiently find the related research reference? Try to search for key words Logic Synthesis, Synthesis, EDA, etc., see which conferences does the research always occur?
\end{enumerate}
\end{problem}
\vspace*{4\baselineskip}
\end{document}